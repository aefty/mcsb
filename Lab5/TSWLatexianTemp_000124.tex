\documentclass[unicode,11pt,a4paper,twoside,numbers=endperiod,openany]{article}

\usepackage[latin1]{inputenc}
\usepackage{url}
\usepackage{graphicx}
\usepackage{epic,eepic}         % Include epic graphs.
\usepackage{subfig}
\usepackage{psfrag}             % Allow TeX in eps files.
%\usepackage{times}
\usepackage{tabularx}
\usepackage{listings}
\usepackage{color}
\usepackage{amsmath}
%\usepackage{mcode}
%\usepackage{wasysym}
%\usepackage{marvosym}
\usepackage{pifont}     % www.tug.org/tex-archive/macros/latex/required/psnfss/
\usepackage{pgfplots}


\definecolor{Brown}{cmyk}{0,0.81,1,0.60}
\definecolor{OliveGreen}{cmyk}{0.64,0,0.95,0.40}
\definecolor{CadetBlue}{cmyk}{0.62,0.57,0.23,0}
\definecolor{Red}{cmyk}{0.0,1.0,0.0,0}
\lstset{
  language=matlab,
  basicstyle=\ttfamily,
  frame=tb,
  framesep=5pt,
  basicstyle=\footnotesize,
  stringstyle=\ttfamily,
  commentstyle=\ttfamily\color{Brown},
  keywordstyle=\ttfamily\color{OliveGreen}\bfseries,
  identifierstyle=\ttfamily,
  tabsize=2,
  showstringspaces=true
}

% --- Paper format -------------------------------------------------------- %

\setlength{\voffset}{0in}
\setlength{\hoffset}{-1in}
\setlength{\topmargin}{0cm}
\setlength{\topskip}{-11.0cm}
\setlength{\parindent}{0pt}

%\setlength{\headsep}{0.0cm}
\setlength{\headheight}{0cm}

\setlength{\oddsidemargin}{1.5cm}
\setlength{\evensidemargin}{1.5cm}

\setlength{\textwidth}{18cm}
\setlength{\textheight}{23cm}
\newcommand{\figwidths}{0.95\textwidth}
\newcommand{\figwidthd}{0.45\textwidth}
\pgfplotsset{compat=1.10}

\author{Aryan Eftekhari // Disscussed with Edoardo Vecchi}

\title{PDE Lab - Assignment 4}
\begin{document}
\maketitle
\newpage









\section{Prework  - Calculations}
To solve for $f$ we will find the laplacian of $u_0$ :

$$ u_0 =x^2y^2(2x-3)(2y-3)$$

$$ -f  =  -((2y^3 - 3y^2)*(12x - 6) + (2x^3 - 3x^2)(12y - 6))$$

\section{Solution of Poisson's equation}
\subsection{Mesh generation}
\lstinputlisting{makeQuadGrid.m}
\newpage










\subsection{Finite Element Code -  Quad}

\lstinputlisting{makeLocalMassMatrix.m}
\lstinputlisting{makeLocalLaplacianMatrix.m}
\lstinputlisting{makeLocalRhs.m}
\lstinputlisting{addBCVNZ.m}
\lstinputlisting{conv.m}
\newpage











\subsection{Performance study}
The performance (time to completion) of the assembly and solution of the Quad elements can be observed in the plot below. 
\\*

Almost all the time is spent in the assembly of the data and matrix. The time to solve the matrix is vastly improved by using the "sparse" matrix command in Matlab, indeed the solution of the matrix takes so little time it is barely visible (black strip above the blue area). With this said the current hindering factor is clearly the matrix assembly time.

\begin{center}
   \includegraphics[scale=1]{timeStud}
\end{center}
\newpage









\subsection{Triangular Grid and Convergence}
From the plot below it can be observed that the triangular element is less accurate than the quad. However, the triangular element continues to decrease in the global error with increasing mesh density - as expected. 
\\*

On the contrary, the quadrilateral elements seem to not increase in accuracy with higher discretization, indeed, the plot shows  erratic behaviour with both $L_2$ and $H_1$ norms fluctuation (note the loglog scale). This behaviour is unexpected but almost immeasurable  in magnitude (order of magnitude $10^{-14}$).
\\*

A possible explanation is the truncation error induced by floating point operations. This hypothesis is coherent with the erratic behaviour of the error and its small magnitude.

\begin{center}
   \includegraphics[scale=1]{conv}
\end{center}
\newpage

\subsection{Finite Element Code -  Triangle}
\lstinputlisting{makeTriGrid.m}
\lstinputlisting{makeLocalMassMatrixT.m}
\lstinputlisting{makeLocalLaplacianMatrixT.m}
\newpage











\begin{figure}
\caption{Visualization Quadrilateral Elements - X[0,1], Y[0,1]}
\begin{tabular}{cc}
  \includegraphics[width=85mm]{quad_4} &   \includegraphics[width=85mm]{quad_80} \\
(a) N = 4 & (b) N=80 \\[6pt]
 \includegraphics[width=85mm]{quad_200} &   \includegraphics[width=85mm]{quad_999} \\
(c) N=200 & (d) Exact Solution \\[6pt]
\end{tabular}
\end{figure}

\begin{figure}
\caption{Visualization Triangular Elements - X[0,1], Y[0,1]}
\begin{tabular}{cc}
  \includegraphics[width=85mm]{tri_4} &   \includegraphics[width=85mm]{tri_80} \\
(a) N = 4 & (b) N=80 \\[6pt]
 \includegraphics[width=85mm]{tri_200} &   \includegraphics[width=85mm]{tri_999} \\
(c) N=200 & (d) Exact Solution \\[6pt]
\end{tabular}

\end{figure}











\end{document}

